%----------------------------------------------------------------------------------------
%	PACKAGES AND THEMES
%----------------------------------------------------------------------------------------
\documentclass[aspectratio=169,xcolor=dvipsnames]{beamer}
\usetheme{SimplePlus}

\usepackage{hyperref}
\usepackage{graphicx} % Allows including images
\usepackage{booktabs} % Allows the use of \toprule, \midrule and \bottomrule in tables
\usepackage{media9}
\usepackage[dvipsnames]{xcolor}
\usepackage[]{multimedia}

\graphicspath{{./images/}}

%----------------------------------------------------------------------------------------
%	TITLE PAGE
%----------------------------------------------------------------------------------------

\title[short title]{Fair Division} % The short title appears at the bottom of every slide, the full title is only on the title page
\subtitle{Cake Cutting Algorithms: Be Fair if You Can}

\author[Iniyan Joseph] {Iniyan Joseph}

\institute[UTD] % Your institution as it will appear on the bottom of every slide, may be shorthand to save space
{
    University of Texas at Dallas % Your institution for the title page
}
\date{} % Date, can be changed to a custom date


%----------------------------------------------------------------------------------------
%	PRESENTATION SLIDES
%----------------------------------------------------------------------------------------
\begin{document}

%\beamerdefaultoverlayspecification{<+->}
\begin{frame}
    % Print the title page as the first slide
    \titlepage
\end{frame}

\begin{frame}[allowframebreaks]{Overview}
    % Throughout your presentation, if you choose to use \section{} and \subsection{} commands, these will automatically be printed on this slide as an overview of your presentation
    \tableofcontents
\end{frame}

%------------------------------------------------
\section{Introduction to Fair Division}
%------------------------------------------------
\begin{frame}{Meeting 1}
	\begin{block}{Agenda}
		\begin{itemize}
			\item Introduction
			\item Fair Division for n Players
			\begin{itemize}
				\item Banach Knaster
				\item Dubins Spanier
				\item Even Paz
			\end{itemize}
		\end{itemize}
	\end{block}
\end{frame}
%------------------------------------------------
\begin{frame}{Introduction}
	Imagine two people want to share this cake.
	\begin{figure}
		\includegraphics[width=0.75\linewidth]{cakeImage}
	\end{figure}
\end{frame}

%------------------------------------------------
\begin{frame}{Introduction}
    \begin{itemize}
        \item The cake is complicated
        \item The two people may value different parts of the cake differently\pause
        \item Can we come up with an algorithm where both people are happy?
    \end{itemize}
\end{frame}
%------------------------------------------------
\section{Cut and Choose}
%------------------------------------------------
\begin{frame}
	\Huge{\centerline{\textbf{Cut and Choose}}}
\end{frame}
%------------------------------------------------
\begin{frame}{Procedure}
	\begin{enumerate}
		\item Player 1 cuts the cake into what they believe is half
		\item Player 2 chooses the piece which they think is better
	\end{enumerate}
\end{frame}
%------------------------------------------------
\begin{frame}{Proof of Correctness}
	\begin{itemize}
		\item Player 1 recieves $\frac{1}{2}$ of the cake
		\item Player 1 values Player 2's allocation to also be worth $\frac{1}{2}$\pause
		\item Player 2 recieved the piece which they thought was better
		\item Player 2 must value their piece to be at least $\frac{1}{2}$ of the cake
	\end{itemize}
\end{frame}
%------------------------------------------------
\section{Fair Division for $n$}
%------------------------------------------------
\subsection{Banach-Knaster Last Diminisher}
%------------------------------------------------
\begin{frame}
	\Huge{\centerline{\textbf{Banach-Knaster Last Diminisher}}}
\end{frame}
%------------------------------------------------
\begin{frame}{Procedure}
	\begin{enumerate}
		\item Player 1 cuts $\frac{1}{n}$ of the cake
		\item Player 2 through n
		\begin{itemize}
			\item If they believe the piece is worth $> \frac{1}{n}$ of the cake, they may trim it
			\item If they believe the piece is worth $\leq \frac{1}{n}$ of the cake, they may pass it to the next person
		\end{itemize} \pause
		\item The last person to trim the piece recieves it and drops out\pause
		\item Repeat until no players remain
	\end{enumerate}
\end{frame}
%------------------------------------------------
\begin{frame}{Proof of Correctness}
	\begin{itemize}
		\item Cutting a piece to be $> \frac{1}{n}$ can cause further division to be limited to $< \frac{1}{n}$ of the cake
		\item This is most easily seen with an extreme example \pause
		\begin{enumerate}
			\item Person 1 cuts 98\% of the cake, with the goal of taking it for themselves.
			\item After passing the cake around, the last diminisher has only cut the piece down to 97\% of the value of the cake
			\item Person 1 now cannot receive more than 3\% of the cake.
		\end{enumerate}
	\end{itemize}
\end{frame}
%------------------------------------------------
\subsection{Dubins-Spanier Moving Knife}
%------------------------------------------------
\begin{frame}
	\Huge{\centerline{\textbf{Dubins-Spanier Moving Knife}}}
\end{frame}
%------------------------------------------------
\begin{frame}{Procedure}
	\begin{itemize}
		\item Rather than having many cuts, a "moving knife" can be used to allocate chunks of cake.
		\begin{enumerate}
			\item A knife moves over the cake continuously from one side to the opposite side (for example from left to right)
			\item When a person thinks that the portion remaining from the starting side/previous cut is worth $\frac{1}{n}$, then they may say "Cut", and they will take the portion on the left side.
		\end{enumerate}
	\end{itemize}
\end{frame}
%------------------------------------------------
\begin{frame}{Proof of Correctness}
	\begin{itemize}
		\item The same person who said "Cut" at any given point would have been the last diminisher in in the Banach-Knaster Last Diminisher Method.\pause
		\item On a surface level, this seems to take n-1 cuts, but this is incorrect. Instead, it takes an infinite number of cuts perpendicular to the direction of movement.
	\end{itemize}
\end{frame}
%------------------------------------------------
\subsection{Even-Paz Divide and Conquer}
%------------------------------------------------
\begin{frame}
	\Huge{\centerline{\textbf{Even-Paz Divide and Conquer}}}
\end{frame}
%------------------------------------------------
\begin{frame}{Procedure}
	\begin{enumerate}
		\item Players 1...n-1 cut the cake in half
		\item Player n compares the cake to the left and to the right of middle cut and chooses the piece which they think is bigger.\pause
		\item Player n and the players on the side n chose repeat the procedure on that side
		\item The remaining players repeat the procedure on the other side
	\end{enumerate}
\end{frame}
%------------------------------------------------
\section{Other Fair and Envy-Free Schemes}
\begin{frame}{Meeting 2}
	\begin{block}{Agenda}
		\begin{itemize}
			\item Stromquist Envy-Free Moving Knife
			\item Austin's Perfect Division for n=2
			\item Aziz-Mackenzie Envy-Free Procedure
		\end{itemize}
	\end{block}
\end{frame}
%------------------------------------------------
\subsection{Stromquist Envy-Free Moving Knife}
%------------------------------------------------
\begin{frame}
\Huge{\centerline{\textbf{Stromquist Envy Free Moving Knife}}}
\end{frame}
%------------------------------------------------
%\begin{frame}{Procedure}
%\end{frame}
%------------------------------------------------
\subsection{Austin's Perfect Division for n=2}
%------------------------------------------------
\begin{frame}
	\Huge{\centerline{\textbf{Austin's Perfect Division for n=2}}}
\end{frame}
%------------------------------------------------
\begin{frame}{Defining Perfect Division}
	Perfect Division is the allocation where
	\[\forall_i V_i(A_i) = \frac{1}{n}\]
	No bounded algorithm exists for perfect division
\end{frame}
%------------------------------------------------
\begin{frame}{Procedure}
	\begin{enumerate}
		\item Knife moves from left to right
		\item A player may call stop \pause
		\item A second knife is placed on the left edge
		\item Both knives move parallely
		\item The other player calls stop \pause
		\item Assign pieces arbitrarily
	\end{enumerate}
\end{frame}
%------------------------------------------------
\subsection{Aziz-Mackenzie Envy-Free Procedure}
%------------------------------------------------
\begin{frame}
	\Huge{\centerline{\textbf{Aziz-Mackenzie Envy-Free Procedure}}}
\end{frame}
%------------------------------------------------
\begin{frame}{Aziz-Mackenzie Envy-Free Procedure for n}
	\centerline{\url{{https://youtu.be/fvM8ow6zNw4?si=AGrOGF7vSZSGt4QK&t=711}}}
\end{frame}
%----------------------------------------------------------------------------------------
\begin{frame}{Meeting 3}
	\begin{block}{Agenda}
		\begin{itemize}
			\item HOWTO: Reading Papers
			\item Unequal Division Na{\"i}vely
			\item Cutting into 1-sized parts
			\item Ramsey Partitions
			\item \textcolor{Gray}{Halving}
		\end{itemize}
	\end{block}
	Next Week: Finish Chapter 3 \& Chapter 4
\end{frame}
%----------------------------------------------------------------------------------------
\section{Paper Reading}
\subsection{Intentions}
\begin{frame}{Reading Papers}
	Let's be honest\\Most papers are dryer than the Sahara Desert\pause
	\newline
	\newline
	So why read papers?
\end{frame}
%----------------------------------------------------------------------------------------
\begin{frame}{Reading Papers}
	\begin{itemize}
		\item Understanding the current research better \pause
		\item To gain background knowledge \pause
		\item To find interesting questions to work on \pause
		\item Because the professor said so \pause
	\end{itemize}
	Fundamentally, we are learning: But what are we trying to learn?
\end{frame}
%----------------------------------------------------------------------------------------
\begin{frame}{Reading Papers}
	But what are we trying to learn?\\\pause
	Our goal when reading a paper is to contextualize that paper's findings in the field.
\end{frame}
%----------------------------------------------------------------------------------------
\subsection{Structure}
\begin{frame}{Reading Papers}
	Thankfully, scientists are aware of this, so they write about it.\pause
	\\Parts of a Paper
	\begin{itemize}
		\item Title \& Authors \pause
		\item Abstract \pause
		\item Introduction \pause
		\item Related Works \pause
		\item Content\pause
		\item Discussion/Conclusion
	\end{itemize}
\end{frame}
%----------------------------------------------------------------------------------------
\subsection{Illustrated Guide to PhD - Matt Might}
\begin{frame}{PhD}
	\url{https://matt.might.net/articles/phd-school-in-pictures/IllustratedGuidePhD-Matt-Might.pdf}
\end{frame}
%----------------------------------------------------------------------------------------
\section{Unequal Division}
\subsection{Anecdote}
\begin{frame}{Dividing Camels}
	\includegraphics[width=0.78\linewidth]{Camel}\\
	First son gets $\frac{1}{2}$. Second son gets $\frac{1}{3}$. Third son gets $\frac{1}{9}$
\end{frame}
%----------------------------------------------------------------------------------------\
\subsection{Na{\"i}ve Method}
\begin{frame}{Na{\"i}ve Method}
	\begin{itemize}
		\item Duplicate each player proportional to their ratio.\pause
		\item We will say that 
		\begin{itemize}
			\item Player 1 recieves $A_1$ allocation
			\item Player 2 recieves $A_2$ allocation
		\end{itemize}\pause
		\item Using the Even-Paz method, $\theta((\sum A)\log(\sum A))$ cuts are required.\newline
		\item We can do better!
	\end{itemize}
\end{frame}
%----------------------------------------------------------------------------------------
\subsection{Cutting Ones}
\begin{frame}{Cutting Ones}
	\begin{itemize}
		\item Similar to the cut-and-choose algorithm\pause
		\item Player 1 divides the cake into 1-sized parts
		\item Player 2 chooses $A_2$ of the 1-sized parts
	\end{itemize}
\end{frame}
%----------------------------------------------------------------------------------------
\subsection{Ramsey}
\begin{frame}{Ramsey Partitions}
	\begin{itemize}
		\item Ramsey theory is simply the study of edge colorings of complete graphs\pause
		\item We can observe this finding $R(3, 3)$\pause
		\item The same property seen in Ramsey Theory can also be used to help partition
	\end{itemize}
\end{frame}
%-------------------------------------------------------------------------------
\begin{frame}{Ramsey Partitions}
	\begin{itemize}
		\item Assume $A_1 < A_2$
		\item Player 1 cuts what they percieve to be $A_1$ of the cake.
		\item The smaller piece can be called $X_1$ and the larger piece can be called $X_2$
		\item If $\mu_2(X_2) < A_2$ (If the person thinks they got less than they were supposed to)
		\item Player 2 takes $X_1$ and division continues in the ratio $A_2 - A_1: A_1$ until no cake remains
	\end{itemize}
\end{frame}
%-------------------------------------------------------------------------------
\begin{frame}{Ramsey Partitions}
	\begin{itemize}
		\item This can be simplified using ramsey partitions
		\item Person 1 cuts the Ramsey Partitions of $\sum A$
		\item Person 2 marks all pieces $\mu_2(X_j) > \mu_1(X_j)$ (Everything they are willing to accept)\pause
		\begin{enumerate}
			\item If the sum of a subset of the marked pieces $= A_2$, we can give those pieces to Player 2 and give the rest to Player 1\pause 
			\item Otherwise, Player 1 may choose pieces summing to $A_1$ of the unmarked pieces. 
			\item This works because of the Ramsey Partitioning!
		\end{enumerate}
	\end{itemize}
\end{frame}
%-------------------------------------------------------------------------------
\begin{frame}{Meeting 4}
	\begin{block}{Agenda}
		\begin{itemize}
			\item Review
			\item Halving
			\item The Serendipity of Disagreement
			\item \textcolor{Gray}{Summer}
		\end{itemize}
	\end{block}
\end{frame}
%-------------------------------------------------------------------------------
\subsection{Halving}
\begin{frame}{Halving}
	\begin{itemize}
		\item Why were ramsey partitions more effective than Cutting Ones? \pause
		\item They allow us to allocate more of the cake at once \pause
		\item We can get rid of more of the cake at once by cutting closer to $\frac{1}{2}$ of the cake at once.
	\end{itemize}
\end{frame}
%-------------------------------------------------------------------------------
\begin{frame}{Halving}
	\begin{itemize}
		\item Take the example of dividing 13 with a 8:5 ratio \pause
		\item Let's draw the tree \pause
		\item It can be shown that this is at least as good as Ramsey Partitioning.
	\end{itemize}
\end{frame}
%-------------------------------------------------------------------------------
\section{Disagreement}
\begin{frame}{Disagreement}
	\begin{itemize}
		\item Sometimes it may feel as if differences of opinion cause conflict
		\item But through the existence of envy-free division, we can see this may actually lead to more social good
	\end{itemize}
\end{frame}
%-------------------------------------------------------------------------------
\begin{frame}{Disagreement}
	\begin{itemize}
		\item Each person divides the cake into n parts (n-1 lines)
		\item Each person can receive a piece of cake which they think is at least $\frac{1}{n}$ \pause
		\item If at least 1 person disagrees, there will be a way to allocate with excess
	\end{itemize}
\end{frame}
%-------------------------------------------------------------------------------
\section{Other Interpretations}
\subsection{Strong Fair Division}
%-------------------------------------------------------------------------------
\begin{frame}{Meeting 5}
	\begin{block}{Agenda}
		\begin{itemize}
			\item The Serendipity of Disagreement
			\item Strong Fair Division
			\item Classes of Fair Division
		\end{itemize}
	\end{block}
\end{frame}
\begin{frame}{Strong Fair Division}
	\begin{itemize}
		\item All players think they got more than $\frac{1}{n}$ of the cake \pause
		\item It is clear to see that there may not always exist a strongly fair allocation \pause
		\item If $ \exists_{i, j} \mu_i \neq \mu_j$, we can always create a strongly fair allocation
	\end{itemize}
This is the value of disagreement!
\end{frame}
%-------------------------------------------------------------------------------
\begin{frame}{Strong Fair Division}
	\begin{itemize}
		\item Imagine a piece of cake $A \subseteq X$
		\item If $\mu_1 (A) > \mu_2 (A)$, we may continue \pause
		\begin{itemize} 
			\item Let $ \mu_1 (A) = a$
			\item Let $ \mu_2 (A) = b$
		\end{itemize}
		\item Let $ X - A = B$
	\end{itemize}
\end{frame}
\begin{frame}{Strong Fair Division$ _{n=2}$}
	\begin{itemize}
		\item $ p < q $ since $ \frac{p}{q} < 1$
		\item $1-b > 1-\frac{p}{q}$ = $\frac{q-p}{q}$ \pause
		\item Player 1 cuts A into $p$ equal parts, each with value > $ \frac{1}{q}$
		\item Player 2 cuts B into $q-p$ equal parts, each with value >  $\frac{1}{q}$ \pause
		\item They each disagreed with the other person initially $ \implies $
		\item $ \exists_{i}~\mu_2(A_i) < \frac{1}{q} \wedge \exists_{j}~\mu_1(B_j) < \frac{1}{q}$ 
		\begin{itemize}
			\item Each of them think the other person got $ < \frac{1}{q}$ for some piece
		\end{itemize} \pause
		\item Cut and choose all other piece of cake ($ X - A_i - B_i $)
		\item Give $ A_i $ to $ 1 $ and $ B_j $ to $ 2 $ \pause
		\item q-1 cuts 
	\end{itemize}
\end{frame}
\begin{frame}{Strong Fair Division$ _{n=3}$}
	\begin{itemize}
		\item We will take a bottom-up approach to adding a $3_{rd}$ person
		\item Without loss of generality, we want $\frac{2k-1}{3k-1}\ast\mu_1(X_1) > \frac{1}{3}$ \pause
		\item Remember what we did for the bottom-up method: We cut the $X_1$ into $3k-1$ equal pieces
		\item Player 3 chooses $k$ pieces
		\begin{itemize}
			\item $ 1 - \frac{2k-1}{3k-1} = \frac{k}{3k-1}$
		\end{itemize}
		\item Player 1 must be content
		\item Player 3 must also be content
	\end{itemize}
\end{frame}
\begin{frame}{Strong Fair Division}
	Let's look at the number of cuts needed - 
	\begin{itemize}
		\item The number of cuts is dependent on k
		\item $\frac{1}{2} + \frac{\epsilon}{2}~~~0 < \epsilon < 1$
		\item We must find $k~~\text{s.t}~\frac{(2k-1)}{2}\ast (1+\epsilon) > \frac{3k-1}{3}$
		\item $\epsilon (6k-1) > 1 $
		\item k can grow to become quite large depending on $\epsilon$. Since we do not know $\epsilon$ beforehand, we must choose a large number of cuts.
	\end{itemize}
\end{frame} % Could we use a punishment scheme to force person n to remove some of their cake do top-down from 1/n? After all, only 1 person thinks that they got > 1/n in last-diminisher
\subsection{Classes of Fair Division}
\begin{frame}{Classes of Fair Division}
	\begin{itemize}
		\item Finite vs Infinite 
		\begin{itemize}
			\item Irrational Unequal Shares
		\end{itemize}\pause
		\item Bounded vs Unbounded (Upper Bounding)\pause
		\item Continuous vs Discrete (Moving Knife?)
	\end{itemize}
\end{frame}
\end{document}
